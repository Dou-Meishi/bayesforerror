%% -*- coding: utf-8 -*-

\section{Numeric Examination for Set A and C}
对于首项近似下的 Set A,
应用 $\dmsdkp$ 的隐式表达式 Eq~\eqref{eq:DOBSetA},
可以得到 PRC2015 TABLE.~II 的结果。
这里我们使用了函数 \inlinecode{scipy.optimize.root_scalar()}
来求得 $p\%>(p\%)_t$ 时的结果
(见~\href{https://github.com/Dou-Meishi/bayesforerror/blob/master/code/tab2.py}{\texttt{tab2.py}}).

对于首项近似下的 Set C,
可以应用公式 Eq~\eqref{eq:PrDelta1} 数值计算得到 $\dmspr(\dmsdeltako|
\dmsccck)$,
然后使用隐式表达式 Eq~\eqref{eq:DOB} 求出 $\dmsdkp$
(见~\href{https://github.com/Dou-Meishi/bayesforerror/blob/master/code/tab3-1.py}{\texttt{tab3-1.py}}).

如果 \autoref{tab:tab2} 中的小数四舍五入到和 PRC2015 TABLE.~III 中相应精度,
那么可以发现两张表的结果是完全一致的,
没有任何不同。

对于不使用首项近似的情形,
可以从 Eq~\eqref{eq:PrDeltah} 出发。
观察发现,
数值计算 $\dmspr(\Delta|\dmsccck)$ 的难点就在于
$\dmspr_h(\Delta|\dmscbar)$,
除此之外都是简单的一重积分。
而只有 $\dmspr_h(\Delta|\dmscbar)$ 是通过对 $\dmsccch$ 的 $h$ 重积分给出的。
好在文献 PRC2017 APPENDIX~A 中给出了 Set A, B, C 对应 $\dmspr(c_n|\dmscbar)$
情形下 $\dmspr_h(\Delta|\dmscbar)$ 的简化表达式,
\begin{equation}
  \label{eq:PrhABC}
  \begin{aligned}
    \dmspr_h^{(A)}(\Delta|\dmscbar) &=
    \frac{1}{2\pi}\int_{-\infty}^{+\infty}
    \cos(t\Delta) \prod_{n=k+1}^{k+h}
    \frac{\sin( Q^nt\dmscbar)}{ Q^nt\dmscbar}\dmsdif t\\
    \dmspr_h^{(B)}(\Delta|\dmscbar) &= \dmspr_h^{(A)}(\Delta|\dmscbar) \\
    \dmspr_h^{(C)}(\Delta|\dmscbar) &=
    \frac{1}{q\dmscbar\sqrt{2\pi}}\exp\Bigl(- \frac{
        \Delta^2}{2q^2\dmscbar^2}\Bigr),
    \qquad q^2\equiv\sum_{n=k+1}^{k+h}Q^{2n}
    = Q^{2k+2} \frac{1-Q^{2h}}{1-Q^2}
  \end{aligned}
\end{equation}
这里 $\dmspr_h^{(A)}(\Delta|\dmscbar)=\dmspr_h^{(B)}(\Delta|\dmscbar)$
是因为 Set A, B 有相同概率分布的 $\dmspr(c_n|\dmscbar)$.

对于 Set C,
将 Eq~\eqref{eq:PrhABC} 代入 Eq~\eqref{eq:PrDeltah},
并令 $x=1/\dmscbar$,
\begin{equation}
  \label{eq:52}
  \dmspr_h^{(C)}(\Delta|\dmsccck) = \frac{1}{q\sqrt{2\pi}}\cdot\frac{
    \int_{1/\dmscbar_>}^{1/\dmscbar_<}x^{n_c}\exp\biggl[- \frac{x^2}{2}\Bigl(
    \dmsccckb^2+\frac{\Delta^2}{q^2}\Bigr)\biggr]\dmsdif x}{
    \int_{1/\dmscbar_>}^{1/\dmscbar_<}x^{n_c-1}\exp\biggl(- \frac{
      x^2}{2}\dmsccckb^2\biggr)\dmsdif x},
  \qquad \dmsccckb^2\equiv\sum_{n=0}^k c_n^2
\end{equation}
利用特殊函数 $\upGamma\,(s,x)$,
可以计算出 Eq~\eqref{eq:52} 中分子分母的积分,
\begin{equation}
  \label{eq:upGammasx}
  \upGamma\,(s,x) = \int_x^\infty t^{s-1}e^{-t}\dmsdif t
\end{equation}
\begin{equation}
  \label{eq:53}
  \int_{1/\dmscbar_>}^{1/\dmscbar_<}x^{n_c-1}e^{-\alpha x^2}\dmsdif x
  = \frac{\alpha^{-n_c/2}}{2}
  \int_{\alpha/\dmscbar_>^2}^{\alpha/\dmscbar_<^2}
    t^{ \frac{n_c}{2}-1}e^{-t}\dmsdif t
  = \frac{\alpha^{-n_c/2}}{2}
  \Biggl[\upGamma\,\biggl( \frac{n_c}{2},
    \frac{\alpha}{\dmscbar_>^2}\biggr)
      - \upGamma\,\biggl( \frac{n_c}{2},
      \frac{\alpha}{\dmscbar_<^2}\biggr)\Biggr]
\end{equation}
\begin{equation}
  \label{eq:54}
  \int_{1/\dmscbar_>}^{1/\dmscbar_<}x^{n_c}e^{-\beta x^2}\dmsdif x
  = \frac{\beta^{-(n_c+1)/2}}{2}
  \int_{\beta/\dmscbar_>^2}^{\beta/\dmscbar_<^2}
    t^{ \frac{n_c+1}{2}-1}e^{-t}\dmsdif t
  = \frac{\beta^{-(n_c+1)/2}}{2}
  \Biggl[\upGamma\,\biggl( \frac{n_c+1}{2},
    \frac{\beta}{\dmscbar_>^2}\biggr)
      - \upGamma\,\biggl( \frac{n_c+1}{2},
      \frac{\beta}{\dmscbar_<^2}\biggr)\Biggr]
\end{equation}
Eq~\eqref{eq:54} 和 Eq~\eqref{eq:53} 的计算非常类似,
仅需要令 $n_c\mapsto n_c+1, \alpha\mapsto\beta$。
这里 $\alpha,\beta$ 定义为
\begin{equation}
  \alpha\equiv \frac{1}{2}\dmsccckb^2,
  \qquad
  \beta\equiv \frac{1}{2}\Bigl(\dmsccckb^2+ \frac{\Delta^2}{q^2}
  \Bigr)
\end{equation}

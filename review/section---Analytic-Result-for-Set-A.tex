%% -*- coding: utf-8 -*-

\section{Analytic Result for Set A}
给定 Set A (See~\ref{tab:priors}) 中的先验概率,
$\dmspr(\dmsdeltako|\dmsccck)$ 可由 Eq~\eqref{eq:PrDelta1} 解析地求出,
并且由于最后的函数形式非常简单,
相应的 $\dmsdkp$ 也可以直接计算 Eq~\eqref{eq:DOB} 中的积分得出。

因为常值函数存在归一化的问题,
此时 $\dmscbar$ 的积分区间实际上取不到 $(0,+\infty)$。
Set A 选取的区间为 $(\dmscbar_<,\dmscbar_>)$
(表现为 $\dmspr(\dmscbar)$ 中的 $\theta$ 函数),
并且假定它足够宽,
\begin{equation}
  \label{eq:4-1}
  \dmscbar_< < \dmscbar_{(k)} < \dmscbar_>,\qquad
  \dmscbar_{(k)}\equiv\max\{|c_0|,|c_1|,\ldots,|c_k|\}
\end{equation}
这样,
Eq~\eqref{eq:PrDelta1} 中分母的积分就容易算出,
\begin{equation}
  \label{eq:4-2}
  \begin{aligned}
    \int_0^\infty\dmspr(\dmsccck|\dmscbar)\dmspr(\dmscbar)\dmsdif\dmscbar
    &=  \int_{\dmscbar_<}^{\dmscbar_>} \frac{1}{\dmscbar\ln\dmscbar_>/\dmscbar_<}
    \prod_{n=0}^{k} \frac{\theta(\dmscbar-|c_n|)}{2\dmscbar}\dmsdif\dmscbar\\
    &=  \frac{1}{2^{k+1}\ln\dmscbar_>/\dmscbar_>}\int_{\dmscbar_{(k)}}^{\dmscbar_>}
    \frac{\dmsdif\dmscbar}{\dmscbar^{k+2}} \\
    &= \frac{1}{k+1} \frac{1}{2^{k+1}\ln\dmscbar_>/\dmscbar_>}\cdot\biggl(
    \frac{1}{\dmscbar^{k+1}}\biggr)\dmsvat[\bigg]{\dmscbar_>}{\dmscbar_{(k)}}
  \end{aligned}
\end{equation}
类似地,
Eq~\eqref{eq:PrDelta1} 中分子的积分计算如下,
\begin{equation}
  \label{eq:4-3}
  \begin{aligned}
    \int_0^\infty\dmspr(c_{k+1}=\dmsdeltako/Q^{k+1}|\dmscbar)
    \dmspr(\dmsccck|\dmscbar)\dmspr(\dmscbar)\dmsdif\dmscbar
    &= \int_{\dmscbar_<}^{\dmscbar_>} \frac{1}{\dmscbar\ln\dmscbar_>/\dmscbar_<}
    \prod_{n=0}^{k+1}\frac{\theta(\dmscbar-|c_n|)}{2\dmscbar}\dmsdif\dmscbar\\
    &= \frac{\theta(\dmscbar_>-\dmscbar_{(k+1)})}{2^{k+2}
      \ln\dmscbar_>/\dmscbar_>}\int^{\dmscbar_>}_{\max(\dmscbar_<,\dmscbar_{(k+1)})}
    \frac{\dmsdif\dmscbar}{\dmscbar^{k+3}} \\
    &= \frac{1}{k+2} \frac{\theta(\dmscbar_>-\dmscbar_{(k+1)})
    }{2^{k+2}\ln\dmscbar_>/\dmscbar_>}\cdot\biggl(
    \frac{1}{\dmscbar^{k+1}}\biggr)\dmsvat[\bigg]{\dmscbar_>}{
      \max(\dmscbar_<,\dmscbar_{(k+1)})}\\
    &\qquad\dmscbar_{(k+1)}\equiv\max(\dmscbar_{(k)},|\dmsdeltako|/Q^{k+1})
  \end{aligned}
\end{equation}

这里 $\theta(\dmscbar_>-\dmscbar_{(k+1)})$ 与
$\max(\dmscbar_<,\dmscbar_{(k+1)})$ 的出现是未曾像 Eq~\eqref{eq:4-1} 那样假设
$\dmscbar_< <\dmscbar_{(k+1)}<\dmscbar_>$ 的结果。
最后,
在将 Eq~\eqref{eq:4-2}, \eqref{eq:4-3} 代入 Eq~\eqref{eq:PrDelta1} 前,
还需将结果中的 $k+1$ 替换成 $n_c$,
因为有时 $c_0,c_1,\ldots,c_k$ 可能并不会全部出现,
所以要用 $n_c$ 来表示这 $k+1$ 个系数中不为零的个数。
总之,
Eq~\eqref{eq:PrDelta1} 可以化为
\begin{equation}
  \label{eq:PrDeltaSetA}
  \begin{aligned}
    \dmspr(\dmsdeltako|\dmsccck) &=
    \frac{\theta(\dmscbar_>-\dmscbar_{(k+1)})}{2Q^{k+1}}\cdot
    \frac{n_c}{n_c+1}\cdot\frac{(1/\dmscbar^{n_c+1})\dmsvat{\dmscbar_>}{
        \max(\dmscbar_<,\dmscbar_{(k+1)})}}{(1/\dmscbar^{n_c})\dmsvat{
        \dmscbar_>}{\dmscbar_{(k)}}} \\
    &= \frac{1}{2Q^{k+1}} \frac{n_c}{n_c+1} \times
    \begin{cases}
      \frac{\frac{1}{\dmscbar^{n_c+1}}
        \dmsvat{\dmscbar_>}{\dmscbar_{(k)}}}{
        \frac{1}{\dmscbar^{n_c}}
        \dmsvat{\dmscbar_>}{\dmscbar_{(k)}}},
      &\quad |\dmsdeltako|\le\dmscbar_{(k)}Q^{k+1} \\
      \frac{\frac{1}{\dmscbar^{n_c+1}}
        \dmsvat{\dmscbar_>}{|\dmsdeltako|/Q^{k+1}}}{
        \frac{1}{\dmscbar^{n_c}}
        \dmsvat{\dmscbar_>}{\dmscbar_{(k)}}},
      &\quad \dmscbar_{(k)}Q^{k+1}<|\dmsdeltako|\le\dmscbar_>Q^{k+1} \\
      0, &\quad |\dmsdeltako|>\dmscbar_>Q^{k+1}
    \end{cases}
  \end{aligned}
\end{equation}
这表明 $\dmspr(\dmsdeltako|\dmsccck)$ 与 $\dmsdeltako$ 的函数关系事实上
非常简单:
在区间 $(-\dmscbar_{(k)}Q^{k+1},+\dmscbar_{(k)}Q^{k+1})$ 里,
取值为与 $\dmsdeltako$ 无关的常数;
在区间 $(-\infty,-\dmscbar_>Q^{k+1})$ 和 $(+\dmscbar_>Q^{k+1},+\infty)$ 里,
取值为零;
在这两种区间之外,
函数值与 $\dmsdeltako$ 的依赖关系也具有 $ax^{-n}+b$ 的简单形式。
这使得直接计算 Eq~\eqref{eq:DOB} 中的积分变得简单,
\begin{equation}
  \label{eq:DOBSetA}
  p\%=
  \begin{cases}
    \frac{\dmsdkp}{Q^{k+1}} \frac{n_c}{n_c+1}
    \frac{ \frac{1}{\dmscbar^{n_c+1}}\dmsvat{\dmscbar_{(k)}}{\dmscbar_>}}{
      \frac{1}{\dmscbar^{n_c}}\dmsvat{\dmscbar_{(k)}}{\dmscbar_>}},
    &\quad \dmsdkp\le\dmscbar_{(k)}Q^{k+1} \\
    1 - \frac{1}{Q^{k+1}} \frac{n_c}{n_c+1}
    \frac{1}{\frac{1}{\dmscbar^{n_c}}\dmsvat{\dmscbar_{(k)}}{\dmscbar_>}}
    \Biggl[ \frac{Q^{(k+1)(n_c+1)}}{n_c}\biggl( \frac{1}{\Delta_k^{n_c}}
      \biggr)\dmsvat[\bigg]{\dmscbar_>Q^{k+1}}{\dmsdkp}-
      \frac{\dmscbar_>Q^{k+1}-\dmsdkp}{\dmscbar_>^{n_c+1}}\Biggr],
    &\quad \dmscbar_{(k)}Q^{k+1} < \dmsdkp \le \dmscbar_>Q^{k+1} \\
    1, &\quad \dmsdkp > \dmscbar_>Q^{k+1}
  \end{cases}
\end{equation}
有时判定条件 $\dmsdkp\le\dmscbar_{(k)}Q^{k+1}$ 会换成
\begin{equation}
  p\%\le(p\%)_{\textrm{t}},
  \qquad (p\%)_{\textrm{t}}\equiv
  \frac{\frac{1}{\dmscbar^{n_c+1}}\dmsvat{\dmscbar_{(k)}}{\dmscbar_>}}{
    \frac{1}{\dmscbar^{n_c}}\dmsvat{\dmscbar_{(k)}}{\dmscbar_>}}
  \cdot \frac{n_c}{n_c+1}\dmscbar_{(k)}
\end{equation}
在 $(\dmscbar_<,\dmscbar_>)\to(0,\infty)$ 的极限情形,
Eq~\eqref{eq:PrDeltaSetA}, \eqref{eq:DOBSetA} 相应地化简为
\begin{equation}
  \dmspr(\dmsdeltako|\dmsccck)= \frac{1}{2Q^{k+1}} \frac{n_c}{n_c+1}
  \frac{1}{\dmscbar_{(k)}}\times
  \begin{cases}
    1,&\quad |\dmsdeltako|\le\dmscbar_{(k)}Q^{k+1}\\
    \biggl( \frac{\dmscbar_{(k)}Q^{k+1}}{|\dmsdeltako|}\biggr)^{n_c+1},
    &\quad \dmsdeltako > \dmscbar_{(k)}Q^{k+1}
  \end{cases}
\end{equation}
\begin{equation}
  \dmsdkp = \dmscbar_{(k)}Q^{k+1}\times
  \begin{cases}
    \frac{n_c+1}{n_c}p\%,
    &\quad p\le \frac{n_c}{n_c+1}\\
    \biggl[ \frac{1}{(n_c+1)(1-p\%)}\biggr]^{1/n_c},
    &\quad p > \frac{n_c}{n_c+1}
  \end{cases}
\end{equation}

%% -*- coding: utf-8 -*-

\section{Numeric Examination for Set A and C}
对于首项近似下的 Set A,
应用 $\dmsdkp$ 的隐式表达式 Eq~\eqref{eq:DOBSetA},
可以得到 PRC2015 TABLE.~II 的结果。
这里我们使用了函数 \inlinecode{scipy.optimize.root_scalar()}
来求得 $p\%>(p\%)_t$ 时的结果
(见~\href{https://github.com/Dou-Meishi/bayesforerror/blob/master/code/tab2.py}{\texttt{tab2.py}}).

对于首项近似下的 Set C,
可以应用公式 Eq~\eqref{eq:PrDelta} 数值计算得到 $\dmspr(\dmsdeltako|
\dmsccck)$,
然后使用隐式表达式 Eq~\eqref{eq:DOB} 求出 $\dmsdkp$
(见~\href{https://github.com/Dou-Meishi/bayesforerror/blob/master/code/tab3-1.py}{\texttt{tab3-1.py}}).

如果 \autoref{tab:tab2} 中的小数四舍五入到和 PRC2015 TABLE.~III 中相应精度,
那么可以发现两张表的结果是完全一致的,
没有任何不同。

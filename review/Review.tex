\documentclass[12pt]{article}
%% -*- coding: utf-8 -*-
%% genearl config of article document class

%% encoding
\usepackage[utf8]{inputenc}
\usepackage[T1]{fontenc}

%% font select
\usepackage[tbtags]{amsmath}
\usepackage{unicode-math}
    \setmathfont{Libertinus Math}
    \setmathfont{XITS Math}[StylisticSet={1},range=cal]
    \unimathsetup{math-style=ISO, bold-style=ISO}
\usepackage[fontset=none,zihao=5,heading=false]{ctex}
    \ctexset{contentsname={Contents},today=old}
    \setCJKmainfont[BoldFont={SimHei}, ItalicFont={STKaiti}]{NSimSun}
    \setCJKsansfont[BoldFont={Microsoft YaHei Bold}]{Microsoft YaHei Light}
    \setCJKmonofont{Microsoft YaHei Light}
    \defaultCJKfontfeatures{Scale=0.962216}
    \setmainfont[BoldFont={El Messiri Bold},
                 SmallCapsFont={Andada SC},
                 ItalicFont={Economica Italic},
                 BoldItalicFont={Economica Italic}]{Libertinus Serif}
    \setsansfont[BoldFont={El Messiri Bold},
                 ItalicFont={Economica Italic},
                 BoldItalicFont={Economica Italic}]{Libertinus Sans}
    \setmonofont{Inconsolata}

%% figure and table
\usepackage{graphicx}
\usepackage{subcaption}
%% \usepackage{wrapfig}
\usepackage{adjustbox}
%% \usepackage{rotating}
\usepackage{booktabs}
%% \usepackage{longtable}
%% \usepackage{colortbl}
%% \usepackage{makecell}
%% \usepackage{diagbox}
\usepackage{multirow}
%% \usepackage{paralist}
%% \usepackage{tabularx}
%% \usepackage{array}
%% \usepackage[labelsep=period, labelfont=bf]{caption}
%% \usepackage[section]{placeins}
%% \usepackage{afterpage}

%% frequently used packages
\usepackage{xcolor}

\usepackage{hyperref}
    \definecolor{linkcolor}{rgb}{0,.145,.698}
    \definecolor{urlcolor}{rgb}{.71,0.21,0.01}
    \definecolor{citecolor}{rgb}{.12,.54,.11}
    \hypersetup{urlcolor=urlcolor,
                linkcolor=linkcolor,
                citecolor=citecolor,
                pagebackref=true,
                menucolor=black,
                letterpaper=true,
                breaklinks=true,
                colorlinks
}

\usepackage[sorting=nyt]{biblatex}
\usepackage{siunitx}
    \sisetup{output-decimal-marker = .,
             per-mode = symbol,
             binary-units = true,
             separate-uncertainty = true
}
\usepackage{geometry}
    \geometry{verbose, a4paper}
    \geometry{left=25mm, right=25mm, top=35mm, bottom=25mm}

\usepackage{fancyhdr}
    \pagestyle{fancy}
    \fancyhf{}
    \makeatletter
    \renewcommand{\title}[1]{\gdef\@title{#1}\gdef\titlestring{#1}}
    \renewcommand{\maketitle}{\newpage\null\thispagestyle{plain}
    \vskip1em\noindent{\Huge\titlestring}\vskip3em}
    \makeatother
    \renewcommand{\headrulewidth}{0pt}
    \fancyhead[L]{\today}
    \fancyhead[RO]{\textsc{\titlestring}\hskip1.5em\arabic{page}}

\usepackage[shortlabels]{enumitem}
    \setlist{noitemsep, topsep=5pt plus 2pt minus 3pt}

%% Misc
\usepackage{microtype}
\usepackage{textcomp}           % for package siunitx
%% \usepackage{xparse}
%% \usepackage{csquotes}
%% \usepackage[toc,page]{appendix}
%% \usepackage[bottom]{footmisc}
%% \usepackage{fancyhdr}
%% \usepackage{lastpage}
%% \usepackage{multicol}
%% \usepackage{fancyhdr}
%% \usepackage{IEEEtrantools}
%% \usepackage{listings}           % try using tocolorbox instead


%% --------------------
%%   CMD DEFINITION
%% --------------------

\renewcommand\arraystretch{1.3} % baselineskip

%% symbol expression
\def\mdis{\displaystyle}
\def\dmsdif{\,\mathrm{d}}
\newcommand*{\dmsNN}{\char"2115\relax}
\newcommand*{\dmsbar}[1]{\mkern1.5mu\overline{\mkern-1.5mu#1\mkern-1.5mu}\mkern1.5mu}

\usepackage{tcolorbox}
\tcbuselibrary{listings, breakable, xparse}

\DeclareTotalTCBox{\inlinecode}{O{python} v}
{verbatim,colback=white,colframe=white}
{\lstinline[language=#1, keywordstyle=\color{cyan}\bfseries]^#2^}

%% \DeclareTCBInputListing{\inputcode}{O{[latex]tex} m !O{}}{
%%  fonttitle=\bfseries, 
%%  colback=yellow!5,colframe=yellow!50!black,listing only,
%%  before skip=0.5cm, after skip=0.5cm,
%%  breakable,left=10mm, listing inputencoding=utf8,
%%  listing file={#2}, listing options={
%%         language={#1},
%%         basicstyle=\scriptsize\ttfamily, 
%%         keywordstyle=\color{orange},       % core keywords
%%         keywordstyle={[2]\color{purple}}, % built-ins
%%         stringstyle=\color{teal}\ttfamily,
%%         commentstyle=\color{gray},
%%         upquote=true,                      % requires textcomp
%%         lineskip=-2pt,
%%         numbers=left,
%%         numberstyle=\small\color{red!75!black}}, #3
%% }

%% \newtcblisting{mycode}[2][python]{%
%%     fonttitle = \Large\bfseries,
%%     colback=yellow!5,colframe=yellow!50!black,listing only,
%%     before skip=0.5cm, after skip=0.5cm, 
%%     breakable,left=10mm,title = #2,
%%     listing options={
%%         language=#1,
%%         basicstyle=\normalsize\ttfamily, 
%%         keywordstyle=\color{orange},       % core keywords
%%         keywordstyle={[2]\color{purple}}, % built-ins
%%         stringstyle=\color{teal}\ttfamily,
%%         commentstyle=\color{gray},
%%         upquote=true,                      % requires textcomp
%%         lineskip=-2pt,
%%         numbers=left,
%%         numberstyle=\small\color{red!75!black}
%% }}

\endinput

\newcommand*\dmspr{\textrm{pr}}
\newcommand*\dmsvat[3][\big]{#1|_{\raise3pt\hbox{$\scriptstyle#2$}}^{\lower3pt\hbox{$\scriptstyle#3$}}}
\def\dmscbar{\dmsbar{c}}
\def\dmsccck{c_0,c_1,\ldots,c_k}
\def\dmsdeltako{\Delta_k^{(1)}}
\def\dmsdkp{d_k^{(p)}}

\title{Review of Some Literature}
\begin{document}
\maketitle
\tableofcontents

\section{Mathematical Expression of the Problem}
假定可观测量 $X$ 有级数展开式 $X=\sum c_nQ^n$,
记 $k$ 阶截断误差为
\begin{equation}
  \Delta_k\equiv\sum_{n=k+1}^\infty c_nQ^n
\end{equation}
欲求 $\dmspr(\Delta_k|\dmsccck)$.

\section{Assumptions and Priors}
\begin{enumerate}[i)]
\item $c_n$ 可视为某个上界 $\dmscbar$ 给定后的随机变量,
  $c_n$ 彼此独立,而且服从同一个概率分布 $\dmspr(c_n|\dmscbar)$.
\item 从不同的先验信息出发可导出 $\dmspr(c_n|\dmscbar)$,
  $\dmspr(\dmscbar)$. \autoref{tab:1} 已列出三种用于讨论。
\end{enumerate}

\section{General Formula of the First-term Approximation}
考虑 $\Delta\approx\Delta_k^{(1)}\equiv c_{k+1}Q^{k+1}$ 的情形,
从等式
\begin{equation}
  \label{eq:1}
  \dmspr(\dmsdeltako|\dmsccck)=
  \int\dmspr(\dmsdeltako|\dmsccck,c_{k+1})
  \dmspr(c_{k+1}|\dmsccck)\mdif c_{k+1}
\end{equation}
出发,
分别计算积分式中的两项如下,
\begin{equation}
  \label{eq:2}
  \begin{aligned}
    \dmspr(\dmsdeltako|\dmsccck,c_{k+1}) &=
    \dmspr(\dmsdeltako|c_{k+1}) \\
    &= \delta(\dmsdeltako-c_{k+1}Q^{k+1})\\
    &= \frac{1}{Q^{k+1}}\delta\biggl(c_{k+1}-
    \frac{\dmsdeltako}{Q^{k+1}}\biggr)
  \end{aligned}
\end{equation}
\begin{equation}
  \label{eq:3}
  \begin{aligned}
    \dmspr(c_{k+1}|\dmsccck) &= \int\dmspr(c_{k+1}|\dmsccck,\dmscbar)
    \dmspr(\dmscbar|\dmsccck)\mdif\dmscbar \\
    &= \int\dmspr(c_{k+1}|\dmscbar) \frac{\dmspr(\dmsccck|\dmscbar)
      \dmspr(\dmscbar)}{\dmspr(\dmsccck)}\mdif\dmscbar \\
    &= \frac{\int\dmspr(c_{k+1}|\dmscbar)\dmspr(\dmsccck|\dmscbar)
      \dmspr(\dmscbar)\mdif\dmscbar}{
    \int\dmspr(\dmsccck|\dmscbar)\dmspr(\dmscbar)\mdif\dmscbar}
  \end{aligned}
\end{equation}
分子分母中的 $\dmscbar$ 是积分变量,并不相同。
此外,
$\dmscbar$ 的积分区间是 $(0,+\infty)$,
而 $c_{k+1}$ 的积分区间时 $(-\infty,+\infty)$。
有时会先将 $\dmscbar$ 的积分区间取为 $(\epsilon,1/\epsilon)$,
计算完成后再令 $\epsilon \to 0^+$.
Eq~\eqref{eq:2}, \eqref{eq:3} 代入 Eq~\eqref{eq:1} 得
\begin{equation}
  \label{eq:PrDelta}
  \dmspr(\dmsdeltako|\dmsccck)=
  \frac{\int\dmspr(c_{k+1}=\dmsdeltako/Q^{k+1}|\dmscbar)
    \dmspr(\dmsccck|\dmscbar)\dmspr(\dmscbar)\mdif\dmscbar}{
    Q^{k+1}\int\dmspr(\dmsccck|\dmscbar)\dmspr(\dmscbar)\mdif\dmscbar},
  \qquad
  \dmspr(\dmsccck|\dmscbar)=\prod_{n=0}^k\dmspr(c_n|\dmscbar)
\end{equation}
连乘式 $\prod\dmspr(c_n|\dmscbar)$ 中的项数可能少于 $k+1$,
因为对于某些可观测量来说,
部分系数可能自动等于零,
不存在是随机变量的说法。
当 $\dmspr(\Delta_k|\dmsccck)$ 求出后,
可以进一步算出 $p\%$ 置信区间 $(-\dmsdkp,+\dmsdkp)$,
\begin{equation}
  \label{eq:DOB}
  p\%=\int_{-\dmsdkp}^{+\dmsdkp}\dmspr(\Delta_k|\dmsccck)\mdif\Delta_k
\end{equation}


\begin{table}
  \caption{Prior pdfs.}
  \label{tab:1}
  \centering
  \begin{tabular}{lccc}
    \toprule
    Set & $\dmspr(c_n|\dmscbar)$ && $\dmspr(\dmscbar)$ \\
    \hline
    A & $\frac{1}{2\dmscbar}\theta(\dmscbar-|c_n|)$ &\qquad&
    $ \frac{1}{\dmscbar\ln\dmscbar_>/\dmscbar_<}\theta(
    \dmscbar-\dmscbar_<)\theta(\dmscbar_>-\dmscbar)$ \\
    B & $\frac{1}{2\dmscbar}\theta(\dmscbar-|c_n|)$ &&
    $ \frac{1}{\dmscbar\sigma\sqrt{2\pi}}\exp\bigl[
    - \frac{(\ln\dmscbar)^2}{2\sigma^2}\bigr]$ \\
    C & $ \frac{1}{\dmscbar\sqrt{2\pi}}\exp\bigl(
    - \frac{c_n^2}{2\dmscbar^2}\bigr)$ &&
    $ \frac{1}{\dmscbar\ln\dmscbar_>/\dmscbar_<}\theta(
    \dmscbar-\dmscbar_<)\theta(\dmscbar_>-\dmscbar)$ \\
    \bottomrule
  \end{tabular}
\end{table}

\section{Analytic Result for Set A}
给定 Set A (See~\ref{tab:1}) 中的先验概率,
$\dmspr(\dmsdeltako|\dmsccck)$ 可由 Eq~\eqref{eq:PrDelta} 解析地求出,
并且由于最后的函数形式非常简单,
相应的 $\dmsdkp$ 也可以直接计算 Eq~\eqref{eq:DOB} 中的积分得出。

因为常值函数存在归一化的问题,
此时 $\dmscbar$ 的积分区间实际上取不到 $(0,+\infty)$。
Set A 选取的区间为 $(\dmscbar_<,\dmscbar_>)$
(表现为 $\dmspr(\dmscbar)$ 中的 $\theta$ 函数),
并且假定它足够宽,
\begin{equation}
  \label{eq:4-1}
  \dmscbar_< < \dmscbar_{(k)} < \dmscbar_>,\qquad
  \dmscbar_{(k)}\equiv\max\{|c_0|,|c_1|,\ldots,|c_k|\}
\end{equation}
这样,
Eq~\eqref{eq:PrDelta} 中分母的积分就容易算出,
\begin{equation}
  \label{eq:4-2}
  \begin{aligned}
    \int_0^\infty\dmspr(\dmsccck|\dmscbar)\dmspr(\dmscbar)\mdif\dmscbar
    &=  \int_{\dmscbar_<}^{\dmscbar_>} \frac{1}{\dmscbar\ln\dmscbar_>/\dmscbar_<}
    \prod_{n=0}^{k} \frac{\theta(\dmscbar-|c_n|)}{2\dmscbar}\mdif\dmscbar\\
    &=  \frac{1}{2^{k+1}\ln\dmscbar_>/\dmscbar_>}\int_{\dmscbar_{(k)}}^{\dmscbar_>}
    \frac{\mdif\dmscbar}{\dmscbar^{k+2}} \\
    &= \frac{1}{k+1} \frac{1}{2^{k+1}\ln\dmscbar_>/\dmscbar_>}\cdot\biggl(
    \frac{1}{\dmscbar^{k+1}}\biggr)\dmsvat[\bigg]{\dmscbar_>}{\dmscbar_{(k)}}
  \end{aligned}
\end{equation}
类似地,
Eq~\eqref{eq:PrDelta} 中分子的积分计算如下,
\begin{equation}
  \label{eq:4-3}
  \begin{aligned}
    \int_0^\infty\dmspr(c_{k+1}=\dmsdeltako/Q^{k+1}|\dmscbar)
    \dmspr(\dmsccck|\dmscbar)\dmspr(\dmscbar)\mdif\dmscbar
    &= \int_{\dmscbar_<}^{\dmscbar_>} \frac{1}{\dmscbar\ln\dmscbar_>/\dmscbar_<}
    \prod_{n=0}^{k+1}\frac{\theta(\dmscbar-|c_n|)}{2\dmscbar}\mdif\dmscbar\\
    &= \frac{\theta(\dmscbar_>-\dmscbar_{(k+1)})}{2^{k+2}
      \ln\dmscbar_>/\dmscbar_>}\int^{\dmscbar_>}_{\max(\dmscbar_<,\dmscbar_{(k+1)})}
    \frac{\mdif\dmscbar}{\dmscbar^{k+3}} \\
    &= \frac{1}{k+2} \frac{\theta(\dmscbar_>-\dmscbar_{(k+1)})
    }{2^{k+2}\ln\dmscbar_>/\dmscbar_>}\cdot\biggl(
    \frac{1}{\dmscbar^{k+1}}\biggr)\dmsvat[\bigg]{\dmscbar_>}{
      \max(\dmscbar_<,\dmscbar_{(k+1)})}\\
    &\qquad\dmscbar_{(k+1)}\equiv\max(\dmscbar_{(k)},|\dmsdeltako|/Q^{k+1})
  \end{aligned}
\end{equation}

这里 $\theta(\dmscbar_>-\dmscbar_{(k+1)})$ 与
$\max(\dmscbar_<,\dmscbar_{(k+1)})$ 的出现是未曾像 Eq~\eqref{eq:4-1} 那样假设
$\dmscbar_< <\dmscbar_{(k+1)}<\dmscbar_>$ 的结果。
最后,
在将 Eq~\eqref{eq:4-2}, \eqref{eq:4-3} 代入 Eq~\eqref{eq:PrDelta} 前,
还需将结果中的 $k+1$ 替换成 $n_c$,
因为有时 $c_0,c_1,\ldots,c_k$ 可能并不会全部出现,
所以要用 $n_c$ 来表示这 $k+1$ 个系数中不为零的个数。
总之,
Eq~\eqref{eq:PrDelta} 可以化为
\begin{equation}
  \label{eq:PrDeltaSetA}
  \begin{aligned}
    \dmspr(\dmsdeltako|\dmsccck) &=
    \frac{\theta(\dmscbar_>-\dmscbar_{(k+1)})}{2Q^{k+1}}\cdot
    \frac{n_c}{n_c+1}\cdot\frac{(1/\dmscbar^{n_c+1})\dmsvat{\dmscbar_>}{
        \max(\dmscbar_<,\dmscbar_{(k+1)})}}{(1/\dmscbar^{n_c})\dmsvat{
        \dmscbar_>}{\dmscbar_{(k)}}} \\
    &= \frac{1}{2Q^{k+1}} \frac{n_c}{n_c+1} \times
    \begin{cases}
      \frac{\frac{1}{\dmscbar^{n_c+1}}
        \dmsvat{\dmscbar_>}{\dmscbar_{(k)}}}{
        \frac{1}{\dmscbar^{n_c}}
        \dmsvat{\dmscbar_>}{\dmscbar_{(k)}}},
      &\quad |\dmsdeltako|\le\dmscbar_{(k)}Q^{k+1} \\
      \frac{\frac{1}{\dmscbar^{n_c+1}}
        \dmsvat{\dmscbar_>}{|\dmsdeltako|/Q^{k+1}}}{
        \frac{1}{\dmscbar^{n_c}}
        \dmsvat{\dmscbar_>}{\dmscbar_{(k)}}},
      &\quad \dmscbar_{(k)}Q^{k+1}<|\dmsdeltako|\le\dmscbar_>Q^{k+1} \\
      0, &\quad |\dmsdeltako|>\dmscbar_>Q^{k+1}
    \end{cases}
  \end{aligned}
\end{equation}
这表明 $\dmspr(\dmsdeltako|\dmsccck)$ 与 $\dmsdeltako$ 的函数关系事实上
非常简单:
在区间 $(-\dmscbar_{(k)}Q^{k+1},+\dmscbar_{(k)}Q^{k+1})$ 里,
取值为与 $\dmsdeltako$ 无关的常数;
在区间 $(-\infty,-\dmscbar_>Q^{k+1})$ 和 $(+\dmscbar_>Q^{k+1},+\infty)$ 里,
取值为零;
在这两种区间之外,
函数值与 $\dmsdeltako$ 的依赖关系也具有 $ax^{-n}+b$ 的简单形式。
这使得直接计算 Eq~\eqref{eq:DOB} 中的积分变得简单,
\begin{equation}
  \label{eq:DOBSetA}
  p\%=
  \begin{cases}
    \frac{\dmsdkp}{Q^{k+1}} \frac{n_c}{n_c+1}
    \frac{ \frac{1}{\dmscbar^{n_c+1}}\dmsvat{\dmscbar_{(k)}}{\dmscbar_>}}{
      \frac{1}{\dmscbar^{n_c}}\dmsvat{\dmscbar_{(k)}}{\dmscbar_>}},
    &\quad \dmsdkp\le\dmscbar_{(k)}Q^{k+1} \\
    1 - \frac{1}{Q^{k+1}} \frac{n_c}{n_c+1}
    \frac{1}{\frac{1}{\dmscbar^{n_c}}\dmsvat{\dmscbar_{(k)}}{\dmscbar_>}}
    \Biggl[ \frac{Q^{(k+1)(n_c+1)}}{n_c}\biggl( \frac{1}{\Delta_k^{n_c}}
      \biggr)\dmsvat[\bigg]{\dmscbar_>Q^{k+1}}{\dmsdkp}-
      \frac{\dmscbar_>Q^{k+1}-\dmsdkp}{\dmscbar_>^{n_c+1}}\Biggr],
    &\quad \dmscbar_{(k)}Q^{k+1} < \dmsdkp \le \dmscbar_>Q^{k+1} \\
    1, &\quad \dmsdkp > \dmscbar_>Q^{k+1}
  \end{cases}
\end{equation}
有时判定条件 $\dmsdkp\le\dmscbar_{(k)}Q^{k+1}$ 会换成
\begin{equation}
  p\%\le(p\%)_{\textrm{t}},
  \qquad (p\%)_{\textrm{t}}\equiv
  \frac{\frac{1}{\dmscbar^{n_c+1}}\dmsvat{\dmscbar_{(k)}}{\dmscbar_>}}{
    \frac{1}{\dmscbar^{n_c}}\dmsvat{\dmscbar_{(k)}}{\dmscbar_>}}
  \cdot \frac{n_c}{n_c+1}\dmscbar_{(k)}
\end{equation}
在 $(\dmscbar_<,\dmscbar_>)\to(0,\infty)$ 的极限情形,
Eq~\eqref{eq:PrDeltaSetA}, \eqref{eq:DOBSetA} 相应地化简为
\begin{equation}
  \dmspr(\dmsdeltako|\dmsccck)= \frac{1}{2Q^{k+1}} \frac{n_c}{n_c+1}
  \frac{1}{\dmscbar_{(k)}}\times
  \begin{cases}
    1,&\quad |\dmsdeltako|\le\dmscbar_{(k)}Q^{k+1}\\
    \biggl( \frac{\dmscbar_{(k)}Q^{k+1}}{|\dmsdeltako|}\biggr)^{n_c+1},
    &\quad \dmsdeltako > \dmscbar_{(k)}Q^{k+1}
  \end{cases}
\end{equation}
\begin{equation}
  \dmsdkp = \dmscbar_{(k)}Q^{k+1}\times
  \begin{cases}
    \frac{n_c+1}{n_c}p\%,
    &\quad p\le \frac{n_c}{n_c+1}\\
    \biggl[ \frac{1}{(n_c+1)(1-p\%)}\biggr]^{1/n_c},
    &\quad p > \frac{n_c}{n_c+1}
  \end{cases}
\end{equation}

\section{Numeric Examination for Set A and C}
对于首项近似下的 Set A,
应用 $\dmsdkp$ 的隐式表达式 Eq~\eqref{eq:DOBSetA},
可以得到 PRC2015 TABLE.~II 的结果。
这里我们使用了函数 \inlinecode{scipy.optimize.root_scalar()}
来求得 $p\%>(p\%)_t$ 时的结果
(见~\href{https://github.com/Dou-Meishi/bayesforerror/blob/master/code/tab2.py}{\texttt{tab2.py}}).

对于首项近似下的 Set C,
可以应用公式 Eq~\eqref{eq:PrDelta} 数值计算得到 $\dmspr(\dmsdeltako|
\dmsccck)$,
然后使用隐式表达式 Eq~\eqref{eq:DOB} 求出 $\dmsdkp$
(见~\href{https://github.com/Dou-Meishi/bayesforerror/blob/master/code/tab3-1.py}{\texttt{tab3-1.py}}).

如果 \autoref{tab:tab2} 中的小数四舍五入到和 PRC2015 TABLE.~III 中相应精度,
那么可以发现两张表的结果是完全一致的,
没有任何不同。

%% -*- coding: utf-8 -*-
%% output of tab2.py (Compared with PRC2015)

\begin{table}
  \renewcommand\arraystretch{1.0} % baselineskip
  \caption{PRC2015 TABLE.~II 的结果复现。$\dmscbar_{(k)}=1$.}
  \label{tab:tab2}
  \centering
  \begin{tabular}{*{15}{l}}
    \toprule\hline
    &\quad& min/max &\quad& $Q$ &\quad& $k=0$ &\quad& $k=1$
    &\quad& $k=2$ &\quad& $k=3$ &\quad& $k=4$ \\

    \hline
    %% ========================================
    \multirow{9}{*}{68\%}

    %% ------------------------------
    && 0.001/1000 &&
    \multirow{3}{*}{0.20} &&
	0.312 && 0.0408 && 0.00725 && 0.00136 && 0.000261
    \\
    && 0.25/4.0 &&
    &&
	0.219 && 0.0389 && 0.00717 && 0.00136 && 0.000261
    \\
    && 0.50/2.0 &&
    &&
	0.181 && 0.035 && 0.00677 && 0.00132 && 0.000257
    \\
    \cline{2-15}

    %% ------------------------------
    && 0.001/1000 &&
    \multirow{3}{*}{0.33} &&
	0.515 && 0.111 && 0.0326 && 0.0101 && 0.00319
    \\
    && 0.25/4.0 &&
    &&
	0.362 && 0.106 && 0.0322 && 0.0101 && 0.00319
    \\
    && 0.50/2.0 &&
    &&
	0.299 && 0.0952 && 0.0304 && 0.00976 && 0.00314
    \\
    \cline{2-15}
    
    %% ------------------------------
    && 0.001/1000 &&
    \multirow{3}{*}{0.50} &&
	0.78 && 0.255 && 0.113 && 0.0531 && 0.0255
    \\
    && 0.25/4.0 &&
    &&
	0.549 && 0.243 && 0.112 && 0.053 && 0.0255
    \\
    && 0.50/2.0 &&
    &&
	0.453 && 0.219 && 0.106 && 0.0514 && 0.0251
    \\
    \cline{2-15}

    \hline
    %% ========================================
    \multirow{9}{*}{95\%}

    %% ------------------------------
    && 0.001/1000 &&
    \multirow{3}{*}{0.20} &&
	1.96 && 0.103 && 0.0137 && 0.00226 && 0.000407
    \\
    && 0.25/4.0 &&
    &&
	0.466 && 0.0773 && 0.0129 && 0.00223 && 0.000406
    \\
    && 0.50/2.0 &&
    &&
	0.292 && 0.0558 && 0.0106 && 0.00203 && 0.000388
    \\
    \cline{2-15}

    %% ------------------------------
    && 0.001/1000 &&
    \multirow{3}{*}{0.33} &&
	3.24 && 0.281 && 0.0615 && 0.0168 && 0.00498
    \\
    && 0.25/4.0 &&
    &&
	0.768 && 0.21 && 0.0578 && 0.0166 && 0.00497
    \\
    && 0.50/2.0 &&
    &&
	0.482 && 0.152 && 0.0478 && 0.015 && 0.00474
    \\
    \cline{2-15}
    
    %% ------------------------------
    && 0.001/1000 &&
    \multirow{3}{*}{0.50} &&
	4.91 && 0.645 && 0.214 && 0.0884 && 0.0398
    \\
    && 0.25/4.0 &&
    &&
	1.16 && 0.483 && 0.201 && 0.0873 && 0.0397
    \\
    && 0.50/2.0 &&
    &&
	0.73 && 0.349 && 0.166 && 0.0792 && 0.0379
    \\
    \cline{2-15}

    \hline\bottomrule
  \end{tabular}
\end{table}

\end{document}

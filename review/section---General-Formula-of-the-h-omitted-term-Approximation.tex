%% -*- coding: utf-8 -*-

\section{General Formula of the h-omitted-term Approximation}
考虑 $\Delta\approx\dmsdeltakh\equiv c_{k+1}Q^{k+1}+c_{k+2}Q^{k+2}+\cdots
+c_{k+h}Q^{k+h}$ 的情形,
从等式
\begin{equation}
  \label{eq:1}
  \dmspr (\Delta|\dmsccck)=
  \int \dmspr(\Delta|\dmscbar,\dmsccck)
  \dmspr(\dmscbar|\dmsccck)\dmsdif\dmscbar
\end{equation}
出发,
分别计算积分式中的两项如下,
\begin{equation}
  \label{eq:2}
  \begin{aligned}
    \dmspr(\Delta|\dmscbar,\dmsccck)
    &= \int \dmspr(\Delta|\dmscbar,\dmsccck,\dmsccch)
    \dmspr(\dmsccch|\dmscbar,\dmsccck)\dmsdifccch \\
    &\approx \int \updelta\Bigl(\Delta-\dmsdeltakh\Bigr)
    \dmspr(\dmsccch|\dmscbar)\dmsdifccch
  \end{aligned}
\end{equation}
\begin{equation}
  \label{eq:3}
    \dmspr(\dmscbar|\dmsccck)
    = \frac{\dmspr(\dmsccck|\dmscbar)\dmspr(\dmscbar)}{\dmspr(\dmsccck)},
    \qquad
    \dmspr(\dmsccck)=\int\dmspr(\dmsccck|\dmscbar^\prime)
    \dmspr(\dmscbar)^\prime\dmsdif\dmscbar^\prime
\end{equation}
%% 注意 Eq~\eqref{eq:3} 计算结果表明 $\dmspr(\Delta|\dmscbar,\dmsccck)$
%% 与 $\dmsccck$ 无关,
注意,
当 $\dmscbar$ 给定时,
$\Delta$ 与 $\dmsccck$ 是相互独立的 (因为 $c_n$ 是相互独立的)。
所以,
\begin{equation}
  \label{eq:4}
  \dmspr(\Delta|\dmscbar,\dmsccck)=\dmspr(\Delta|\dmscbar)
\end{equation}
用 $\dmsccch$ 将 Eq~\eqref{eq:4} 两边展开也能看出此结论。
Eq~\eqref{eq:2} 中 $\approx$ 的出现是因为使用了近似
$\Delta\approx\dmsdeltakh$,
所以由此算出来的结果 $\dmspr(\Delta|\dmsccck),\dmspr(\Delta|\dmscbar)$
相应地标记为 $\dmspr_h(\Delta|\dmsccck),\dmspr_h(\Delta|\dmscbar)$。
Eq~\eqref{eq:1}, \eqref{eq:2}, \eqref{eq:3} 中所有 $\dmscbar$ 的积分区间都是
 $(0,+\infty)$,
而 $\dmsccch$ 的积分区间都是 $(-\infty,+\infty)$。
有时会先将 $\dmscbar$ 的积分区间取为 $(\epsilon,1/\epsilon)$,
计算完成后再令 $\epsilon \to 0^+$.

当 $h=1, \Delta\approx\dmsdeltako$,
即使用首项近似时,
Eq~\eqref{eq:2}, \eqref{eq:3} 代入 Eq~\eqref{eq:1} 得
\begin{equation}
  \label{eq:PrDelta1}
  \dmspr_1(\dmsdeltako|\dmsccck)=
  \frac{\int\dmspr(c_{k+1}=\dmsdeltako/Q^{k+1}|\dmscbar)
    \dmspr(\dmsccck|\dmscbar)\dmspr(\dmscbar)\dmsdif\dmscbar}{
    Q^{k+1}\int\dmspr(\dmsccck|\dmscbar)\dmspr(\dmscbar)\dmsdif\dmscbar},
  \qquad
  \dmspr(\dmsccck|\dmscbar)=\prod_{n=0}^k\dmspr(c_n|\dmscbar)
\end{equation}
连乘式 $\prod\dmspr(c_n|\dmscbar)$ 中的项数可能少于 $k+1$,
因为对于某些可观测量来说,
部分系数可能自动等于零,
不存在是随机变量的说法。
当 $\dmspr(\Delta_k|\dmsccck)$ 求出后,
可以进一步算出 $p\%$ 置信区间 $(-\dmsdkp,+\dmsdkp)$,
\begin{equation}
  \label{eq:DOB}
  p\%=\int_{-\dmsdkp}^{+\dmsdkp}\dmspr(\Delta_k|\dmsccck)\dmsdif\Delta_k
\end{equation}

如果不假定 $h=1$,
那么 Eq~\eqref{eq:PrDelta1} 变为
\begin{equation}
  \label{eq:PrDeltah}
  \dmspr_h(\Delta|\dmsccck) = \frac{
    \int \dmspr_h(\Delta|\dmscbar)\dmspr(\dmsccck|\dmscbar)
    \dmspr(\dmscbar)\dmsdif\dmscbar}{
    \int \dmspr(\dmsccck|\dmscbar)\dmspr(\dmscbar)\dmsdif\dmscbar},
  \qquad
  \dmspr_h(\Delta|\dmscbar)=\int \updelta\Bigl(\Delta-\dmsdeltakh\Bigr)
  \prod_{n=k+1}^{k+h}\dmspr(c_n|\dmscbar)\dmsdif c_n
\end{equation}
其中 $\dmspr_h(\Delta|\dmscbar)$ 由 Eq~\eqref{eq:4}, \eqref{eq:3} 得出。
